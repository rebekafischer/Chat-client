\documentclass[12pt, letterpaper]{article}
\title{\textbf{\huge{Chat Client}} \\ Entwicklung eines p2p Chat Clients auf Basis von klassischen Client-Server Protokollen}
\author{\\Rebeka Fischer \\ \emph{3.Semester - IT Infrastruktur} \\ FOM Hochschule für Ökonomie und Management}
\date{06.02.2024}
\begin{document}
\maketitle
\thispagestyle{empty}
\newpage
\setcounter{page}{1}
\tableofcontents
\newpage
\section{Einleitung}
Im Rahmen dieser Arbeit soll ein peer to peer Chat client entwickelt werden, um innerhalb einer Location übers Netzwerks zu kommunizieren. 
Ziel ist es einen funktionierenden Chatroom mit direkter Nachrichtenübertragung ohne Verzögerung zu bauen.
Es handelt sich hierbei um einen einzigen Chatroom für alle Teilnehmer, private chats sind nicht möglich. 
Der Chat client beschränkt sich auf das Versenden und Empfangen von Nachrichten, Funktionen wie Audionachrichten oder Bilder versenden sind nicht Bestandteil. 
Das Verschicken und Zustellen von Nachrichten soll möglich sein ohne einen externen Server zu brauchen. 
Somit sollen die Nachrichten direkt an die Geräte der anderen Teilnehemr geschickt werden. 
Dadurch ergibt sich die Frage: Ist es möglich mit Client-Server-Technologien einen peer to peer Chat Client zu entwickeln?
\section{Grundlagen}
Der Chat client setzt sich im wesentlichen zusammen aus:
\begin{itemize}
    \item Receiver
    \item Transmitter
    \item DiscoveryService
    \item ConfigData
    \item Message types
\end{itemize} 
Beim Starten des Chat clients sucht der DiscoveryService mithilfe eines Multicasts nach anderen Benutzern.
Anschließend wird eine Startmessage mit eigenem Namen und IP Adresse an die IP Adresse des gefundenen Teilnehmers geschickt.
Der Receiver nimmt die Antwort auf die Startmessage entgegen. Diese enthält den Namen des gefundenen Teilnehmers, sodass er passend zur IP Adresse in einer Userliste eingetragen werden kann.
Jeder neue Teilnehmer tritt über den Startprozess dem Chatroom bei. 
\\
\\
Beim Versenden einer Nachricht werden die IP Adressen der User in der ConfigData nachgeschaut. Die Nachricht wird an jede einzelne IP Adresse der Liste geschickt. 
\\
\\
Verlässt ein Teilnehmer den Chatroom, wird eine Exitmessage geschickt. Hierbei wird der Name und die IP Adresse geschickt um aus der Userliste gelöscht zu werden.
Die Anwendung ist in Python geschrieben. Es wird eine REST API für das Versenden von Nachrichten verwendet. http, python 
\section{Umsetzung}
\section{Ergebnisse}
\section{Fazit}
\end{document}