\section{Einleitung}
Im Rahmen dieser Arbeit soll ein peer to peer Chat client entwickelt werden, der die Kommunikation in einer Gruppe im lokalen Netzwerk ermöglicht. 
Die Arbeit richtete sich an ein fachkundiges Publikum und setzt beim Leser ein grundlegendes Verständnis für Computernetzwerke voraus. 
Das Hauptziel besteht darin, einen funktionierenden Chatroom mit direkter Nachrichtenübertragung zu bauen.
Es handelt sich hierbei um einen einzigen Chatroom für alle Teilnehmer, private Chats zwischen einzelnen Nutzern sind nicht möglich. 
Die Chat Anwendung beschränkt sich auf das Versenden und Empfangen von Nachrichten, Funktionen wie das Versenden von Audionachrichten oder Bilder sind nicht Bestandteil. 
Das Verschicken und Zustellen von Nachrichten, sowie das Auffinden, Beitreten und Verlassen des Chats sollen ohne zusätzliche Infrastrukturkomponenten, wie ein externer Server, möglich sein.
Die Nachrichten werden direkt an die Geräte der anderen Teilnehemr im Netzwerk geschickt. 
Angesichts der weit verbreiteten Verfügbarkeit von qualitativ hochwertigen Frameworks für client-server Applikationen und der vergleichsweise geringen Anzahl an peer-to-peer Frameworks mit ähnlicher Güte stellt sich die Frage: 
Ist es möglich mit Client-Server-Technologien einen peer to peer Chat Client zu entwickeln?