\section{Fazit}
Im Rahmen dieser Arbeit wurde ein peer-to-peer Chat client mit client-server Technologie erstellt. 
Nach der theoretischen Ausarbeitung der Funktionsweise der Chatapplikation wurde eine konkrete Umsetzung in Python geschrieben um zu testen,
ob eine praktische Anwendung möglich ist.
Das Programm stellt die Grundfunktionen einer Chatapplikation dar, kann jedoch noch erweitert werden. 
Momentan liegt der Fokus auf dem Austausch von Nachrichten, es könnte ergänzt werden durch Funktionen wie Bilder oder Emojis versenden. Zudem werden
die Nachrichten nicht gespeichert, mit Verlassen des Chats wird auch der Chatverlauf gelöscht und kann nicht bei erneutem Beitritt wiederhergestellt werden.
Des Weiteren könnte man den Exitprozess verbessern. Die Dauer des Austretens aus dem Chatroom erhöht sich, je mehr Teilnehmer vorhanden sind.
Der Ablauf könnte verbessert werden, damit bei einer hohen Tielnhemerzahl das Verlassen und Abmelden schneller ausgeführt wird.
\\
Für eine temporäre Kommunikation innerhalb des selben Netzwerks ist die peer-to-peer Methode für einen Chatroom geeignet, allerdings ist von einem 
öffentlichen Gebrauch abzuraten. Je größer die Teilnehmermenge des Chatrooms wird, desto unübersichtlicher wird der Chatverlauf und es besteht derzeit keine 
Funktion private Chats zu öffnen, sowie einen Verlauf zu speichern.
Zusätzlich dürfte kaum Bedarf in Firmen oder im privaten Leben bestehen, einen einzigen großen Chatroom zu haben, der nur den reinen Nachrichtenaustausch ermöglicht.
Dennoch eignet sich die Anwendung für eine kleine Gruppe, wie beispielsweise Studenten in einer Bibliothek oder einem Hörsaal, um sich schnell über ein Thema auszutauschen.