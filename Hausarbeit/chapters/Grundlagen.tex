\section{Grundlagen}
Für die Entwicklung einer Chat-Anwendung gibt es verschiedene Möglichkeiten zur technischen Umsetzung.
Häufig wird die Client-Server Architektur verwendet. Diese ist das am häufigsten genutze Architekturmodell. 
Der Computer nimmt die Rolle des Clients ein und fragt beim Server Daten an. 
Der Server ist normalerweise ein leistungsstarker Computer, der Daten speichert und sie auf Anfrage zurück gibt.
Client und Server sind über das Netzwerk miteinander verbunden. 
Bei einem Chat client finden zwei Prozesse für die Kommunikation statt.
Zuerst sendet der Client eine Nachricht über das Netzwerk an den Server und wartet auf eine Antwortnachricht. 
Wenn der Server den request des Clients erhält, gibt er eine Antwort mit den geforderten Daten zurück. 
\cite{tanenbaum96}
\\
\\
Dem Client-Server-Modell entgegen steht das Peer-To-Peer-Modell.
Einzelne Computer treten direkt über das Netzwerk miteinander in Kontakt und können kommunizieren.
Es werden Pakete an User geschickt, die als Teil der vorgesehenen Gruppe identifiziert wurden. 
Somit ist jeder Computer ein peer, er kann als Client gegenüber anderen Geräten agieren, aber auch als Server.
Durch das Einnehmen der Client- und Serverfunktion eines jeden Computers ist kein seperater Server notwendig.
Es werden nur die einzelnen Endgeräte benötigt und eine Netzwerkverbindung.
\cite{tanenbaum96}
%was für übertragungsmodi, was unterschied p2p client server, wie machen kann, wie man was baut, an frage annähern  